\chapter{Solutions for Chapter 3}

\ex{3.1}
\begin{circuit}{fig:3.1.1}{JFET current source}
    (0,0) node[njfet] (Q1) {$Q_1$}
    (Q1.D) to[generic, l=load] ++(0,2)
    (Q1.S) to[R, l=$R_\text{S}$] ++(0,-2) coordinate(RS)
    node[ground] {}
    (RS) to[short] (RS-|Q1.G)
    (Q1.G) to[short] (Q1.G|-RS)
\end{circuit}

From Figure 3.21 of the book, one can see that a drain current equal to \SI{1}{\milli\ampere} corresponds to a gate-source voltage of \SI{-0.6}{\volt}.
Therefore:
\[\mans{R_\text{S}=\frac{\SI{0.6}{\volt}}{\SI{1}{\milli\ampere}}=\SI{600}{\ohm}}\]

\ex{3.2}
At $V_\text{GS}=V_\text{G0}$:
\[r_\text{GS}=r_\text{G0}=\frac{1}{2k\left(V_\text{G0}-V_\text{th}\right)}\]
The ratio between $r_\text{DS}$ and $R_\text{G0}$ returns:
\[\mans{\frac{r_\text{DS}}{r_\text{G0}}=\frac{2k\left(V_\text{G0}-V_\text{th}\right)}{2k\left(V_\text{GS}-V_\text{th}\right)}}\]

\ex{3.3}
Being $g_\text{m}$ the differential conductance of the FET operated in aturation region, it can be expressed as:
\[g_\text{m}=\frac{\partial I_\text{D}}{\partial V_\text{GS}}=\frac{\partial}{\partial V_\text{GS}}k\left(V_\text{GS}-V_\text{th}\right)^2=2k\left(V_\text{GS}-V_\text{th}\right)\]
Therefore:
\[\mans{g_\text{m}=\frac{1}{r_\text{DS}}}\]

\ex{3.4}
\begin{enumerate}
    \item The voltage across the drain-gate capacitance when the JFET is switched on ($V_\text{DS}=\SI{0}{\volt}$) is equal to \SI{50}{\volt}-\SI{10}{\volt}=\SI{40}{\volt}. Considering a maximum current across this capacitance equal to \SI{1}{\milli\ampere}:
    \[\mans{t_\text{ON}=\frac{\SI{40}{\volt}\,\SI{200}{\pico\farad}}{\SI{1}{\milli\ampere}}=\SI{8}{\micro\second}}\]
    \item Since the current is equal to the charge over time, we have:
    \[\mans{t_\text{ON}=\frac{\SI{40}{\nano\coulomb}}{\SI{1}{\milli\ampere}}=\SI{40}{\micro\second}}\]
\end{enumerate}

\ex{3.5}
The \SI{1}{\pico\farad} drain-source capacitance happens to be in series with the \SI{10}{\kilo\ohm} load resistance. The capacitive reactance is:
\[X_\text{DS}=\frac{1}{2\pi\SI{1}{\mega\hertz}\,\SI{1}{\pico\farad}}=\SI{160}{\kilo\ohm}\]
Therefore, the feedthrough is given by:
\[\mans{20\log_{10}\frac{\SI{10}{\kilo\ohm}}{\SI{10}{\kilo\ohm}+\SI{160}{\kilo\ohm}}=\SI{-25}{\decibel}}\]

\ex{3.6}
In this case, the output \SI{10}{\kilo\ohm} resistance is in parallel with the \SI{50}{\ohm} $R_\text{ON}$ resistance. Their equivalent resistance is about \SI{50}{\ohm}. Similarly to the previous exercise, the feedthorugh is given by:
\[\mans{20\log_{10}\frac{\SI{50}{\ohm}}{\SI{50}{\ohm}+\SI{160}{\kilo\ohm}}=\SI{-70}{\decibel}}\]

\ex{3.7}
\begin{circuit}{fig:3.7.1}{Zero ohm $R_\text{ON}$}
    (0,0) node[ocirc] {}
    node[above] {in}
    (0,0) to[R,l=$R_\text{S}$] ++(2,0) coordinate(Rs)
    to[C,l=$C_\text{D}$] ++(0,-2)
    node[ground] {}
    (Rs) -- ++(2,0) coordinate(out)
    to[C,l=$C_\text{S}$] ++(0,-2) 
    node[ground] {}
    (out) -- ++(1,0)
    node[ocirc] {}
    node[above] {out}
\end{circuit}

\begin{circuit}{fig:3.7.2}{\SI{75}{\ohm} $R_\text{ON}$}
    (0,0) node[ocirc] {}
    node[above] {in}
    (0,0) to[short] ++(1,0) coordinate(Rs)
    to[C,l=$C_\text{D}$] ++(0,-2)
    node[ground] {}
    (Rs) to[R,l=$R_\text{ON}$] ++(2.5,0) coordinate(out)
    to[C,l=$C_\text{S}$] ++(0,-2) 
    node[ground] {}
    (out) -- ++(1,0)
    node[ocirc] {}
    node[above] {out}
\end{circuit}

For this exercise we assume that the load resistance of \SI{100}{\kilo\ohm} does not load the circuit. 
\begin{enumerate}
    \item The circuit is that of Figure \ref{fig:3.7.1}. Since $C_\text{D}=C_\text{S}=C_\text{T}=\SI{8}{\pico\farad}$, there is a single pole at the frequency $f_\text{p}$:
    \[\mans{f_\text{p}=\frac{1}{4\pi R_\text{S}C_\text{T}}\approx\SI{1}{\mega\hertz}}\]
    \item In this case the circuit is depicted in Figure \ref{fig:3.7.2}. The circuit has one pole at DC and another pole at $f_\text{p}$:
    \[\mans{f_\text{p}=\frac{1}{2\pi R_\text{ON}C_\text{T}}\approx\SI{265}{\mega\hertz}}\]
\end{enumerate}

\ex{3.8}
\begin{circuit}{fig:3.8.1}{OFF-OFF}
    (0,0) node[ocirc] {}
    node[above] {in}
    (0,0) to[R,l=$R_\text{S}$] ++(2,0) coordinate(Rs)
    to[C,l_=$C_\text{C}$] ++(0,-2)
    to[C,l=$C_\text{DS}$] ++(2,0)
    (Rs) to[C,l=$C_\text{DS}$] ++(2,0)
    to[C,l=$C_\text{C}$] ++(0,-2) 
    -- ++(1,0)
    node[ocirc] {}
    node[above] {out}
    to[R,l=$R_\text{out}$] ++(0,-2)
    node[ground] {}
\end{circuit}
\begin{circuit}{fig:3.8.2}{OFF-ON}
    (0,0) node[ocirc] {}
    node[above] {in}
    (0,0) to[R,l=$R_\text{S}$] ++(2,0) coordinate(Rs)
    to[C,l_=$C_\text{C}$] ++(0,-2)
    to[short] ++(2,0)
    (Rs) to[C,l=$C_\text{DS}$] ++(2,0)
    to[C,l=$C_\text{C}$] ++(0,-2) 
    -- ++(1,0)
    node[ocirc] {}
    node[above] {out}
    to[R,l=$R_\text{out}$] ++(0,-2)
    node[ground] {}
\end{circuit}
\begin{circuit}{fig:3.8.3}{ON-OFF}
    (0,0) node[ocirc] {}
    node[above] {in}
    (0,0) to[R,l=$R_\text{S}$] ++(2,0) coordinate(Rs)
    to[C,l_=$C_\text{C}$] ++(0,-2)
    to[C,l=$C_\text{DS}$] ++(2,0)
    (Rs) to[short] ++(2,0)
    to[C,l=$C_\text{C}$] ++(0,-2) 
    -- ++(1,0)
    node[ocirc] {}
    node[above] {out}
    to[R,l=$R_\text{out}$] ++(0,-2)
    node[ground] {}
\end{circuit}
\begin{circuit}{fig:3.8.4}{ON-ON}
    (0,0) node[ocirc] {}
    node[above] {in}
    (0,0) to[R,l=$R_\text{S}$] ++(2,0) coordinate(Rs)
    to[C,l_=$C_\text{C}$] ++(0,-2)
    to[short] ++(2,0)
    (Rs) to[short] ++(2,0)
    to[C,l=$C_\text{C}$] ++(0,-2) 
    -- ++(1,0)
    node[ocirc] {}
    node[above] {out}
    to[R,l=$R_\text{out}$] ++(0,-2)
    node[ground] {}
\end{circuit}

\begin{enumerate}
\item In this case the reference circuit is depicted in Figure \ref{fig:3.8.1}. The cross-coupling is given by:
\[\mans{20\log_{10}\frac{R_\text{out}}{R_\text{out}+R_\text{S}+0.5(X_\text{C}+X_\text{DS})}=\SI{-10.75}{\decibel}}\]
being $X_\text{C}=\frac{1}{2\pi f\,C_\text{C}}=\SI{320}{\kilo\ohm}$ and $X_\text{DS}=\frac{1}{2\pi f\,C_\text{DS}}=\SI{160}{\kilo\ohm}$
\item In this case the reference circuit is depicted in Figure \ref{fig:3.8.2}. The cross-coupling is given by:
\[\mans{20\log_{10}\frac{R_\text{out}}{R_\text{out}+R_\text{S}+\frac{X_\text{C}(X_\text{C}+X_\text{DS})}{2X_\text{C}+X_\text{DS}}}=\SI{-9.6}{\decibel}}\]
\item In this case the reference circuit is depicted in Figure \ref{fig:3.8.3}. The cross-coupling is the same as before
\item In this case the reference circuit is depicted in Figure \ref{fig:3.8.4}. The cross-coupling is given by:
\[\mans{20\log_{10}\frac{R_\text{out}}{R_\text{out}+R_\text{S}+0.5X_\text{C}}=\SI{-8.6}{\decibel}}\]
\end{enumerate}
\todoex{3.9}
\todoex{3.10}
\todoex{3.11}
\todoex{3.12}
\todoex{3.13}
\todoex{3.14}
\todoex{3.15}
\todoex{3.16}
\todoex{3.17}
\todoex{3.18}