\chapter{Solutions for Chapter 3}

\ex{3.1}
\begin{circuit}{fig:3.1.1}{JFET current source}
    (0,0) node[njfet] (Q1) {$Q_1$}
    (Q1.D) to[generic, l=load] ++(0,2)
    (Q1.S) to[R, l=$R_\text{S}$] ++(0,-2) coordinate(RS)
    node[ground] {}
    (RS) to[short] (RS-|Q1.G)
    (Q1.G) to[short] (Q1.G|-RS)

\end{circuit}

From Figure 3.21 of the book, one can see that a drain current equal to \SI{1}{\milli\ampere} corresponds to a gate-source voltage of \SI{-0.6}{\volt}.
Therefore:
\[\mans{R_\text{S}=\frac{\SI{0.6}{\volt}}{\SI{1}{\milli\ampere}}=\SI{600}{\ohm}}\]

\ex{3.2}
At $V_\text{GS}=V_\text{G0}$:
\[r_\text{GS}=r_\text{G0}=\frac{1}{2k\left(V_\text{G0}-V_\text{th}\right)}\]
The ratio between $r_\text{DS}$ and $R_\text{G0}$ returns:
\[\mans{\frac{r_\text{DS}}{r_\text{G0}}=\frac{2k\left(V_\text{G0}-V_\text{th}\right)}{2k\left(V_\text{GS}-V_\text{th}\right)}}\]

\ex{3.3}
Being $g_\text{m}$ the differential conductance of the FET operated in aturation region, it can be expressed as:
\[g_\text{m}=\frac{\partial I_\text{D}}{\partial V_\text{GS}}=\frac{\partial}{\partial V_\text{GS}}k\left(V_\text{GS}-V_\text{th}\right)^2=2k\left(V_\text{GS}-V_\text{th}\right)\]
Therefore:
\[\mans{g_\text{m}=\frac{1}{r_\text{DS}}}\]
\todoex{3.4}
\todoex{3.5}
\todoex{3.6}
\todoex{3.7}
\todoex{3.8}
\todoex{3.9}
\todoex{3.10}
\todoex{3.11}
\todoex{3.12}
\todoex{3.13}
\todoex{3.14}
\todoex{3.15}
\todoex{3.16}
\todoex{3.17}
\todoex{3.18}